%-------------------------
% Резюме на ЛаТеХ
% Автор: Джейк Гутьеррес
% Основано на: https://github.com/sb2nov/resume
% Лицензия: MIT
%------------------------

\documentclass[letterpaper,11pt]{article}

\usepackage{latexsym}
\usepackage[empty]{fullpage}
\usepackage{titlesec}
\usepackage{marvosym}
\usepackage[usenames,dvipsnames]{color}
\usepackage{verbatim}
\usepackage{enumitem}
\usepackage[hidelinks]{hyperref}
\usepackage{fancyhdr}
\usepackage[russian]{babel}
\usepackage{tabularx}
\input{glyphtounicode}


%----------ШРИФТОВЫЕ НАСТРОЙКИ----------
% без засечек
% \usepackage[sfdefault]{FiraSans}
% \usepackage[sfdefault]{roboto}
% \usepackage[sfdefault]{noto-sans}
% \usepackage[default]{sourcesanspro}

% с засечками
% \usepackage{CormorantGaramond}
% \usepackage{charter}


\pagestyle{fancy}
\fancyhf{} % очистить все заголовки и колонтитулы
\fancyfoot{}
\renewcommand{\headrulewidth}{0pt}
\renewcommand{\footrulewidth}{0pt}

% Настройка полей
\addtolength{\oddsidemargin}{-0.5in}
\addtolength{\evensidemargin}{-0.5in}
\addtolength{\textwidth}{1in}
\addtolength{\topmargin}{-.5in}
\addtolength{\textheight}{1.0in}

\urlstyle{same}

\raggedbottom
\raggedright
\setlength{\tabcolsep}{0in}

% Форматирование разделов
\titleformat{\section}{
  \vspace{-4pt}\scshape\raggedright\large
}{}{0em}{}[\color{black}\titlerule \vspace{-5pt}]

% Обеспечить читаемость и разборчивость PDF
\pdfgentounicode=1

%-------------------------
% Пользовательские команды
\newcommand{\resumeItem}[1]{
  \item\small{
    {#1 \vspace{-2pt}}
  }
}

\newcommand{\resumeSubheading}[4]{
  \vspace{-2pt}\item
    \begin{tabular*}{0.97\textwidth}[t]{l@{\extracolsep{\fill}}r}
      \textbf{#1} & #2 \\
      \textit{\small#3} & \textit{\small #4} \\
    \end{tabular*}\vspace{-7pt}
}

\newcommand{\resumeSubSubheading}[2]{
    \item
    \begin{tabular*}{0.97\textwidth}{l@{\extracolsep{\fill}}r}
      \textit{\small#1} & \textit{\small #2} \\
    \end{tabular*}\vspace{-7pt}
}

\newcommand{\resumeProjectHeading}[2]{
    \item
    \begin{tabular*}{0.97\textwidth}{l@{\extracolsep{\fill}}r}
      \small#1 & #2 \\
    \end{tabular*}\vspace{-7pt}
}

\newcommand{\resumeSubItem}[1]{\resumeItem{#1}\vspace{-4pt}}

\renewcommand\labelitemii{$\vcenter{\hbox{\tiny$\bullet$}}$}

\newcommand{\resumeSubHeadingListStart}{\begin{itemize}[leftmargin=0.15in, label={}]}
\newcommand{\resumeSubHeadingListEnd}{\end{itemize}}
\newcommand{\resumeItemListStart}{\begin{itemize}}
\newcommand{\resumeItemListEnd}{\end{itemize}\vspace{-5pt}}

%-------------------------------------------
%%%%%%  РЕЗЮМЕ НАЧИНАЕТСЯ ЗДЕСЬ  %%%%%%%%%%%%%%%%%%%%%%%%%%%%


\begin{document}

%----------ЗАГОЛОВОК----------
\begin{center}
    \textbf{\Huge \scshape Оралбаев Максат} \\ \vspace{1pt}
    \small +7707-393-580 $|$ \href{mailto:code.datum@gmail.com}{\underline{code.datum@gmail.com}} $|$ 
    \href{https://www.linkedin.com/in/maxoralbay/}{\underline{linkedin.com/in/maxoralbay}} $|$
    \href{https://github.com/maxoralbay}{\underline{github.com/maxoralbay}}
\end{center}


%-----------ОБРАЗОВАНИЕ-----------
\section{Образование}
  \resumeSubHeadingListStart
    \resumeSubheading
      {Региональный социально-инновационный университет}{Шымкент, Казахстан}
      {Бакалавр, Учитель физики }{2014-2016}
    \resumeSubheading
      {Инженерно-педагогический университет Дружбы народов}{Шымкент, Казахстан}
      {Бакалавр, Радиотехника, электроника и телекоммуникации }{2007-2011}
  \resumeSubHeadingListEnd


%-----------ОПЫТ РАБОТЫ-----------
\section{Опыт работы}
  \resumeSubHeadingListStart
   \resumeSubheading
      {Full-stack developer}{Август 2023 -- Настоящее время}
      {openTrade}{the UK, London. Удаленно}
      \resumeItemListStart
        \resumeItem{Разработал торгового бота для автоматизации сделок, используя CCXT и pandas-ta.}
        \resumeItem{Реализовал обработку данных через WebSocket API для получения рыночной информации в реальном времени.}
        \resumeItem{Создал крипто-трекер для мониторинга портфеля и анализа рыночных данных.}
        \resumeItem{Работал со смарт-контрактами и интеграцией блокчейна с использованием Web3.py.}
        \resumeItem{Разработал и оптимизировал стратегии на основе технического анализа в Pinescript для платформы TradingView.}
      \resumeItemListEnd

      \resumeSubheading
      {Full-stack developer}{Сентябрь 2023 -- Ноябрь 2024 год}
      {LLC Isker group}{Частичная занятость | Удаленно}
      \resumeItemListStart
        \resumeItem{Разработал сложные Telegram-боты, используя Python и Telegram API.}
        \resumeItem{Оптимизировал базу данных Postgres, написал сложные SQL-запросы, используя SQLAlchemy для упрощения взаимодействия с базой данных.}
        \resumeItem{Участвовал в разработке REST API с использованием Django REST Framework (DRF), внедрив механизм автоматической сериализации данных, что сократило время разработки новых модулей на 25\%.}
        \resumeItem{Настроил обработку очередей через RabbitMQ.}
        \resumeItem{Спроектировал ETL-процессы для извлечения, трансформации и загрузки данных.}
        \resumeItem{Выполнил парсинг и логирование данных, обеспечив корректность и полноту данных в аналитических системах.}
      \resumeItemListEnd

    \resumeSubheading
    {Веб-разработчик}{Август 2020 -- Июнь 2023}
    {Devex Service}{Санкт-Петербург, Россия}
    \resumeItemListStart
      \resumeItem{Full-Stack Backend-разработка: Поддерживал и улучшал сайты с использованием JavaScript, Python, PHP, а также разрабатывал микросервисы на базе Tarantool и Lua, снижая операционные затраты за счет оптимизации систем баз данных и кэширования.}
      \resumeItem{Микросервисы и контейнеризация: Разрабатывал и управлял микросервисами в Docker для сайтов и приложений компании с использованием Laravel, WordPress API, Django и платформ No-Code.}
      \resumeItem{Оптимизация фронтенд-разработки: Руководил разработкой фронтенда на основе NextJS и Vuex, улучшая производительность интерфейсов и вовлеченность пользователей.}
    \resumeItemListEnd


    \resumeSubheading
      {Full-stack developer}{Июль 2019 -- Март 2020}
      {Perfect systems LLC}{Москва, Россия}
      \resumeItemListStart
        \resumeItem{Руководил командой разработки системы парсинга, собирающей и анализирующей отзывы о врачах и медицинских учреждениях в США.}
        \resumeItem{Использовал библиотеку Web Scrapy для автоматизированного сбора данных с различных веб-сайтов.}
        \resumeItem{Реализовал обход защиты сайтов с помощью ротации прокси.}
        \resumeItem{Организовал доставку собранных данных в Django через очереди, обеспечив эффективную обработку и хранение информации.}
      \resumeItemListEnd

      \resumeSubheading
      {Full-stack developer}{Июнь 2017 -- Май 2019}
      {Evrika LLC}{Шымкент, Казахстан}
      \resumeItemListStart
        \resumeItem{Взаимодействие с клиентами и анализ требований: Регулярное взаимодействие с клиентами для понимания их ключевых целей и требований, что обеспечивает соответствие всех веб-ресурсов их потребностям.}
        \resumeItem{Программирование и разработка: Написание и внедрение программ, необходимых для функциональности и поддержки веб-сайтов и систем управления, работа с задачами как на фронтенде, так и на бэкенде.}
        \resumeItem{Поддержка сайтов и решение проблем: Управление и обслуживание сайтов, обеспечение их оптимальной работы и решение возникающих проблем.}
      \resumeItemListEnd
  \resumeSubHeadingListEnd


%-----------МЯГКИЕ НАВЫКИ-----------
\section{Навыки}
 \begin{itemize}[leftmargin=0.15in, label={}]
    {Управление проектами, Тайм-менеджмент, Коммуникация, Адаптивность, Решение проблем, Навыки работы в команде, Креативность}
 \end{itemize}
%-----------ТЕХНИЧЕСКИЕ НАВЫКИ-----------
\section{Технические навыки}
\begin{itemize}[leftmargin=0.15in, label={}]
    \small{\item{
        \textbf{Языки программирования}{: Python (версии 3.x), PHP, SQL, Lua, PineScript, JavaScript, HTML/CSS, Bash} \\
        \textbf{Фреймворки}{: Django, Flask, FastAPI, Django REST Framework (DRF), Laravel, Nuxt.js, Vue, React и др.} \\
        \textbf{СУБД}{: PostgreSQL, MySQL, Redis, MongoDB} \\
        \textbf{DevOps}{: Docker, Kubernetes, Jenkins, GitHub, GitLab CI/CD, GCP, VS Code, SourceTree и др.} \\
        \textbf{Библиотеки}{: Pandas, NumPy, SQLAlchemy, Pytest, CCXT, pandas-ta, Web3.py, Celery, Redis, aiohttp, асинхронные библиотеки и др.} \\
        \textbf{ETL-процессы}{: Создание и поддержка для обработки данных} \\
        \textbf{Алготрейдинг}{: Разработка торговых стратегий и ботов (PineScript, Alpaca API, WebSocket API)} \\
        \textbf{Web3 и NFT}{: Работа со смарт-контрактами, криптотрекерами, интеграция блокчейна} \\
        \textbf{Логирование, обработка и парсинг данных}{: Оптимизация процессов для аналитических задач} \\
        \textbf{Софт-скиллы}{: Работа в команде, ревью кода, декомпозиция задач}
    }}
\end{itemize}
%-------------------------------------------
%-----------Хобби-----------
\section{Хобби}
\begin{itemize}[leftmargin=0.15in, label={}]
    \small{\item{
        Интерес к высшей математике и финансовой математике.
    }}
\end{itemize}

\end{document}
-------------------------------------------